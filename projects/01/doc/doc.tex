% Created 2015-02-08 Sun 23:40
\documentclass[a4paper]{article}
\usepackage[utf8]{inputenc}
\usepackage[T1]{fontenc}
\usepackage{fixltx2e}
\usepackage{graphicx}
\usepackage{longtable}
\usepackage{float}
\usepackage{wrapfig}
\usepackage{rotating}
\usepackage[normalem]{ulem}
\usepackage{amsmath}
\usepackage{textcomp}
\usepackage{marvosym}
\usepackage{wasysym}
\usepackage{amssymb}
\usepackage{hyperref}
\tolerance=1000
\setlength\parindent{0pt}
\usepackage{titling}
\addtolength{\topmargin}{-1.375in}
\addtolength{\textheight}{1.75in}
\addtolength{\oddsidemargin}{-.375in}
\addtolength{\evensidemargin}{-.875in}
\addtolength{\textwidth}{0.75in}
\author{Michael Hunsinger}
\date{\today}
\title{Project One}
\hypersetup{
  pdfkeywords={},
  pdfsubject={},
  pdfcreator={Emacs 24.4.1 (Org mode 8.2.10)}}
\begin{document}

\maketitle
\tableofcontents


\section*{Project}
\label{sec-1}
The first project for Linguistic Geometry, Spring semester of 2015.

\subsection*{Installation}
\label{sec-1-1}
The project was done using Node.js and requires Node.js along with npm to be
installed (usually comes with Node.js). Node.js can be downloaded from the
\href{https://nodejs.org}{Node.Js} website. It can also be install through a package manager if running
Linux.

\subsection*{Build}
\label{sec-1-2}
Once Node.js is installed, navigate to the directory containing the
project. The node modules were not bundled and will have to be installed via
the following command

\begin{verbatim}
npm install
\end{verbatim}

\subsection*{Run}
\label{sec-1-3}
After the modules are installed, run the following command to run the
project.

\begin{verbatim}
npm run project res/chess.json
\end{verbatim}

\subsection*{Input}
\label{sec-1-4}
The \verb~chess.json~ file provided contains all chess pieces. Below is a small
snippet of the file.

\begin{verbatim}
{
    "board": {
	"xMax": 8,
	"yMax": 8,
	"zMax": null,
	"obstacles": [ ]
    },
    "pieces": [
	{
	    "piece": "King",
	    "xCoordinate": 2,
	    "yCoordinate": 7,
	    "reachability": [
		{
		    "condition-1": "| y1 - x1 | <= 1"
		},
		{
		    "condition-1": "| y2 - x2 | <= 1"
		}
	    ]
	}
    ]
}
\end{verbatim}

As you can see, it contains information about the board and any number of
pieces.

\subsubsection*{Board}
\label{sec-1-4-1}
Below are some notes regarding the fields for the board.

\begin{itemize}
\item The board is 1-based and assumes the min is `1' and thus not required.
\item Obstacles are an array of points that cannot be reached by any
piece. Below is a sample of an obstacle object that would cause the point
(3,2) to be unreachable..

\begin{verbatim}
{
    "x": 3,
    "y": 2
}
\end{verbatim}
\item The board information stays constant when reading each piece.
\end{itemize}

\subsubsection*{Pieces}
\label{sec-1-4-2}
Below are some notes regarding the fields for pieces.

\begin{itemize}
\item \verb~x~ and \verb~y~ are the starting coordinates for the given piece.
\item Reachability is an array of conditions that are used to determine if a
given cell is reachable. For each reachability, all conditions must be met
in order for it to be reachable. However, not all reachabilities must be
hold true for the cell to be reachable.
\end{itemize}

\subsection*{Output}
\label{sec-1-5}
Below is sample output for a King.

\begin{verbatim}
7 7 7 7 7 7 7 7 7 7 7 7 7 7 7
7 6 6 6 6 6 6 6 6 6 6 6 6 6 7
7 6 5 5 5 5 5 5 5 5 5 5 5 6 7
7 6 5 4 4 4 4 4 4 4 4 4 5 6 7
7 6 5 4 3 3 3 3 3 3 3 4 5 6 7
7 6 5 4 3 2 2 2 2 2 3 4 5 6 7
7 6 5 4 3 2 1 1 1 2 3 4 5 6 7
7 6 5 4 3 2 1 0 1 2 3 4 5 6 7
7 6 5 4 3 2 1 1 1 2 3 4 5 6 7
7 6 5 4 3 2 2 2 2 2 3 4 5 6 7
7 6 5 4 3 3 3 3 3 3 3 4 5 6 7
7 6 5 4 4 4 4 4 4 4 4 4 5 6 7
7 6 5 5 5 5 5 5 5 5 5 5 5 6 7
7 6 6 6 6 6 6 6 6 6 6 6 6 6 7
7 7 7 7 7 7 7 7 7 7 7 7 7 7 7
\end{verbatim}
% Emacs 24.4.1 (Org mode 8.2.10)
\end{document}
